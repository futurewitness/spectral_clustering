\documentclass[11pt]{article}

\usepackage{epsfig}
\usepackage{amsfonts}
\usepackage{amssymb}
\usepackage{amstext}
\usepackage{amsmath}
\usepackage{xspace}
\usepackage{theorem}
\usepackage{graphicx}
% \usepackage{layout}% if you want to see the layout parameters
% and now use \layout command in the body

% This is the stuff for normal spacing
\makeatletter
 \setlength{\textwidth}{6in}
 \setlength{\oddsidemargin}{0in}
 \setlength{\evensidemargin}{0.5in}
 \setlength{\topmargin}{0in}
 \setlength{\textheight}{9in}
 \setlength{\headheight}{0pt}
 \setlength{\headsep}{0pt}
 \setlength{\marginparwidth}{59pt}

 \setlength{\parindent}{0pt}
 \setlength{\parskip}{5pt plus 1pt}
 \setlength{\theorempreskipamount}{5pt plus 1pt}
 \setlength{\theorempostskipamount}{0pt}
 \setlength{\abovedisplayskip}{8pt plus 3pt minus 6pt}


\newenvironment{proof}{{\bf Proof:  }}{\hfill\rule{2mm}{2mm}}
\newenvironment{proofof}[1]{{\bf Proof of #1:  }}{\hfill\rule{2mm}{2mm}}
\newenvironment{proofofnobox}[1]{{\bf#1:  }}{}
\newenvironment{example}{{\bf Example:  }}{\hfill\rule{2mm}{2mm}}


\newtheorem{theorem}{Theorem}
\newtheorem{lemma}[theorem]{Lemma}
\newtheorem{proposition}[theorem]{Proposition}
\newtheorem{claim}[theorem]{Claim}
\newtheorem{corollary}[theorem]{Corollary}
\newtheorem{definition}[theorem]{Definition}

% math notation
\newcommand{\R}{\ensuremath{\mathbb R}}
\newcommand{\F}{\ensuremath{\mathbb F}}

% vectors
\renewcommand{\vec}[1]{\ensuremath{\mathbf{#1}}}

\newenvironment{sol}
    {\emph{Solution:}
    }

%%%%%%%%%%%%%%%%%%%%%%%%%%%%%%%%%%%%%%%%%%%%%%%%%%%%%%%%%%%%%%%%%%%%%%%%%%%
% Document begins here %%%%%%%%%%%%%%%%%%%%%%%%%%%%%%%%%%%%%%%%%%%%%%%%%%%%
%%%%%%%%%%%%%%%%%%%%%%%%%%%%%%%%%%%%%%%%%%%%%%%%%%%%%%%%%%%%%%%%%%%%%%%%%%%


\newcommand{\headings}{
\large{\textbf{YOUR NAME GOES HERE \hfill 21-241 Fall 2019}\\
\textbf{Homework 3 \hfill Due Friday, September 13}}\\
\rule[0.1in]{\textwidth}{0.01in}
%\thispagestyle{empty}
}

\pagestyle{empty}

\begin{document}

\headings

\begin{enumerate}
\section*{Required Problems}

\item (Strang 2.5.7) If $A$ has row 1 + row 2 = row 3, show that $A$ is not invertible:
\begin{enumerate}
\item Explain why $A\vec{x} = (0,0,1)$ cannot have a solution.  Add eqn 1 + eqn 2.
\item Which right sides $(b_1, b_2, b_3)$ might allow a solution to $A\vec{x} = \vec{b}$?
\item In elimination, what happens to equation 3?
\end{enumerate}

\begin{sol}
Write your solution here.
\end{sol}
\clearpage

\item (Strang 2.5.8) If $A$ has column 1 + column 2  = column 3, show that $A$ is not invertible:
\begin{enumerate}
\item Find a nonzero solution $\vec{x}$ to $A\vec{x} = \vec{0}$.  The matrix is 3 by 3.
\item Elimination keeps column 1 + column 2 = column 3.  Explain why there is no third pivot.
\end{enumerate}

\begin{sol}
Write your solution here.
\end{sol}
\clearpage


\item (Strang 2.5.25) Find $A^{-1}$ and $B^{-1}$ (if they exist) by elimination on $[A | I]$ and $[B | I]$:

\[A= \begin{bmatrix} 2 & 1 & 1 \\ 1 & 2 & 1 \\ 1 & 1 & 2 \end{bmatrix} \text{ and } B=  \begin{bmatrix} 2 & -1 & -1 \\ -1 & 2 & -1 \\ -1 & -1 & 2 \end{bmatrix} \]



\begin{sol}
Write your solution here.
\end{sol}
\clearpage


\item (Strang 3.1.10) Which of the following subsets of $\R^3$ are actually subspaces?
\begin{enumerate}
\item The plane of vectors $(b_1, b_2, b_3)$ with $b_1 = b_2$.
\item The plane of vectors with $b_1 = 1$.
\item The vectors with $b_1b_2b_3 = 0$.
\item All linear combinations of $\vec{v} = (1,4,0)$ and $\vec{w} = (2,2,2)$.
\item All vectors that satisfy $b_1 + b_2 + b_3 = 0$.
\item All vectors with $b_1 \le b_2 \le b_3$.
\end{enumerate}


\begin{sol}
Write your solution here.
\end{sol}
\clearpage


\item (Strang 3.1.19) Describe the column spaces (lines or planes) of these particular matrices:

\[A=\begin{bmatrix} 1 & 2 \\ 0 & 0 \\ 0 & 0 \end{bmatrix} \text{ and } B= \begin{bmatrix} 1 & 0 \\ 0 & 2 \\ 0 & 0 \end{bmatrix} \text{ and } C= \begin{bmatrix} 1 & 0 \\ 2 & 0 \\ 0 & 0 \end{bmatrix}. \]


\begin{sol}
Write your solution here.
\end{sol}
\clearpage


\item (Strang 3.1.23) The columns of $AB$ are combinations of the columns of $A$. This means: \textit{The column space of $AB$ is contained in (possibly equal to) the column space of $A$}. Give an example where the column spaces of $A$ and $AB$ are not equal.


\begin{sol}
Write your solution here.
\end{sol}
\clearpage


\item Let $U$ and $V$ be two subspaces of $\R^n$.
\begin{enumerate}
\item Prove that the intersection $U \cap V$ is also a subspace of $R^n$.
\item Prove that if neither subspace contains the other, then the union $U \cup V$ is not a subspace of $\R^n$
\end{enumerate}


\begin{sol}
Write your solution here.
\end{sol}
\clearpage



\section*{Optional Problems}

\item (Strang 2.5.33)  Find and check the inverses (assuming they exist) of these block matrices:

\[\begin{bmatrix} I & 0 \\ C & I \end{bmatrix} \quad \begin{bmatrix} A & 0 \\ C & D \end{bmatrix} \quad \begin{bmatrix} 0 & I \\ I & D \end{bmatrix} \]

\item (Strang 3.1.23) If we add an extra column $\vec{b}$ to a matrix $A$, then the column space gets larger unless \hspace{1in}.  Give an example where the column space gets larger and an example where it doesn't. Why is $A\vec{x} = \vec{b}$ solvable exactly when the column space \textit{doesn't} get larger--it is the same for $A$ and $[A| \vec{b}]$?

\item Prove that if $A$ is an invertible matrix, that $(A^T)^{-1} = (A^{-1})^T$.

\item Let $M_n$ denote the set of $n \times n$ matrices with entries in the real numbers.
\begin{enumerate}
\item Prove that, with usual rules of matrix addition and scalar multiplication, $M_n$ is a vector space.  What is its dimension?
\item Prove that the set of upper triangular matrices is a subspace of $M_n$.  What is its dimension?
\item Prove that the set of diagonal matrices is a subspace of $M_n$.  What is its dimension?
\end{enumerate}

\item Prove that in the vector space $\F_2^n$, there are exactly $2^k$ vectors in the span of any $k$ linearly independent vectors.



\end{enumerate}




\end{document}