\documentclass[11pt]{article}

\usepackage{epsfig}
\usepackage{amsfonts}
\usepackage{amssymb}
\usepackage{amstext}
\usepackage{amsmath}
\usepackage{xspace}
\usepackage{theorem}
\usepackage{graphicx}
\usepackage{tikz}
\usepackage{pgfplots}

% \usepackage{layout}% if you want to see the layout parameters
% and now use \layout command in the body

% This is the stuff for normal spacing
\makeatletter
 \setlength{\textwidth}{6in}
 \setlength{\oddsidemargin}{0in}
 \setlength{\evensidemargin}{0.5in}
 \setlength{\topmargin}{0in}
 \setlength{\textheight}{9in}
 \setlength{\headheight}{0pt}
 \setlength{\headsep}{0pt}
 \setlength{\marginparwidth}{59pt}

 \setlength{\parindent}{0pt}
 \setlength{\parskip}{5pt plus 1pt}
 \setlength{\theorempreskipamount}{5pt plus 1pt}
 \setlength{\theorempostskipamount}{0pt}
 \setlength{\abovedisplayskip}{8pt plus 3pt minus 6pt}


\newenvironment{proof}{{\bf Proof:  }}{\hfill\rule{2mm}{2mm}}
\newenvironment{proofof}[1]{{\bf Proof of #1:  }}{\hfill\rule{2mm}{2mm}}
\newenvironment{proofofnobox}[1]{{\bf#1:  }}{}
\newenvironment{example}{{\bf Example:  }}{\hfill\rule{2mm}{2mm}}


\newtheorem{theorem}{Theorem}
\newtheorem{lemma}[theorem]{Lemma}
\newtheorem{proposition}[theorem]{Proposition}
\newtheorem{claim}[theorem]{Claim}
\newtheorem{corollary}[theorem]{Corollary}
\newtheorem{definition}[theorem]{Definition}

% math notation
\newcommand{\R}{\ensuremath{\mathbb R}}
\newcommand{\Z}{\ensuremath{\mathbb Z}}
\newcommand{\N}{\ensuremath{\mathbb N}}
\newcommand{\F}{\ensuremath{\mathbb F}}

\newcommand{\size}[1]{\ensuremath{\left|#1\right|}}
\newcommand{\ceil}[1]{\ensuremath{\left\lceil#1\right\rceil}}
\newcommand{\floor}[1]{\ensuremath{\left\lfloor#1\right\rfloor}}


% anupam's abbreviations
\newcommand{\mnote}[1]{\normalmarginpar \marginpar{\tiny #1}}


% vectors
\renewcommand{\vec}[1]{\ensuremath{\mathbf{#1}}}

\newenvironment{sol}
    {\emph{Solution:}
    }


%%%%%%%%%%%%%%%%%%%%%%%%%%%%%%%%%%%%%%%%%%%%%%%%%%%%%%%%%%%%%%%%%%%%%%%%%%%
% Document begins here %%%%%%%%%%%%%%%%%%%%%%%%%%%%%%%%%%%%%%%%%%%%%%%%%%%%
%%%%%%%%%%%%%%%%%%%%%%%%%%%%%%%%%%%%%%%%%%%%%%%%%%%%%%%%%%%%%%%%%%%%%%%%%%%


\newcommand{\headings}{
\large{\textbf{YOUR NAME GOES HERE \hfill 21-241 Fall 2019}\\
\textbf{Homework 4 \hfill Due Friday, September 20}}\\
\rule[0.1in]{\textwidth}{0.01in}
%\thispagestyle{empty}
}

\pagestyle{empty}

\begin{document}

\headings

\begin{enumerate}
\section*{Required Problems}
 \item (Strang 3.2.32, see Page 144 for picture and matrix $A$) Kirchoff's Current Law $A^T\vec{y} = \vec{0}$ says that \textit{current in = current out} at every node.  At node 1 this is $y_3 = y_1 + y_4$.  Write the four equations for Kirchoff's Law at the four nodes (arrows show the positive direction of each $y$). Reduce $A^T$ to $R$ and find three special solutions in the nullspace of $A^T$ (4 by 6 matrix).
 
 \begin{sol}
Write your solution here.
\end{sol}
\clearpage

 \item (from Strang 3.3.1)  Find the RREF for the matrix $A$.  Then describe the column space and null space for $A$.  Finally, describe a complete solution to $A \vec{x} = \vec{b}$.
 
 \[ A = \begin{bmatrix}2 & 4 & 6 & 4 \\ 2 & 5 & 7 & 6 \\ 2 & 3 & 5 & 2 \end{bmatrix}, \quad 
 \vec{b} = \begin{bmatrix} b_1 \\ b_2 \\ b_3 \end{bmatrix} = \begin{bmatrix} 4 \\ 3 \\ 5 \end{bmatrix}\]
 
 \begin{sol}
Write your solution here.
\end{sol}
\clearpage
 
 \item (Strang 3.3.6) What conditions on $b_1$, $b_2$, $b_3$, $b_4$ make each system solvable?  Find $\vec{x}$ in that case:

 \[  \begin{bmatrix} 1 & 2 \\ 2 & 4 \\ 2 & 5 \\ 3 & 9 \end{bmatrix}
 \begin{bmatrix} x_1 \\ x_2 \end{bmatrix}=
 \begin{bmatrix} b_1 \\ b_2 \\ b_3 \\ b_4 \end{bmatrix}, \quad 
\begin{bmatrix} 1 & 2 & 3\\ 2 & 4 & 6 \\ 2 & 5 & 7\\ 3 & 9 & 12\end{bmatrix}
 \begin{bmatrix} x_1 \\ x_2 \\ x_3 \end{bmatrix}=
 \begin{bmatrix} b_1 \\ b_2 \\ b_3 \\ b_4 \end{bmatrix}\] 



 \begin{sol}
Write your solution here.
\end{sol}
\clearpage

 
 \item (Strang 3.4.1) Show that $\vec{v}_1, \vec{v}_2, \vec{v}_3$ are independent but $\vec{v}_1, \vec{v}_2, \vec{v}_3, \vec{v}_4$ are dependent:
\[ \vec{v}_1 = \begin{bmatrix} 1 \\ 0 \\ 0 \end{bmatrix}, 
\vec{v}_2 = \begin{bmatrix} 1 \\ 1 \\ 0 \end{bmatrix} ,
\vec{v}_3 = \begin{bmatrix} 1 \\ 1 \\ 1 \end{bmatrix} ,
\vec{v}_4 = \begin{bmatrix} 2 \\ 3 \\ 4 \end{bmatrix} .\]
Solve $c_1\vec{v}_1 + c_2\vec{v}_2 + c_3 \vec{v}_3 + c_4\vec{v}_4 = \vec{0}$ or $A\vec{x} = \vec{0}$.  The $\vec{v}$'s go in the columns of $A$.
 
\begin{sol}
Write your solution here.
\end{sol}
\clearpage


\item (Strang 3.5.23) $U$ comes from $A$ by subtracting row 1 from row 3:

\[A= \begin{bmatrix} 1 & 3 & 2 \\ 0 & 1 & 1 \\ 1 & 3 & 2 \end{bmatrix} \text{ and } 
U= \begin{bmatrix} 1 & 3 & 2 \\ 0 & 1 & 1 \\ 0 & 0 & 0 \end{bmatrix}\]

Find bases for the two column spaces.  Find bases for the two row spaces. Find bases for the two nullspaces.  Which spaces stay fixed in elimination?

\begin{sol}
Write your solution here.
\end{sol}
\clearpage

 \item Let $A$ be a matrix and $\vec{x}$ and $\vec{y}$ be vectors. 
 \begin{enumerate}
 \item  Prove that $A\vec{x} = A\vec{y}$ if and only if $(\vec{x} -\vec{y}) \in N(A)$.
\item Use Part (a) to explain why if $A\vec{x}_p = \vec{b}$, then any other solution to the matrix equation $A\vec{x} = \vec{b}$ has the form $\vec{x} = \vec{x}_p + \vec{x}_n$, where $\vec{x}_n$ is in $N(A)$.
\end{enumerate}

\begin{sol}
Write your solution here.
\end{sol}
\clearpage


\item In this problem we'll carefully prove an important property of rank.
\begin{enumerate}
\item Let $a$ be a scalar. Prove that $\{\vec{w}_1, \vec{w}_2, \ldots, \vec{w}_{m}\}$ is a linearly independent set of vectors if and only if the set $\{\vec{w}_1 + a\vec{w}_2, \vec{w}_2, \ldots, \vec{w}_{m}\}$ is also linearly independent.
\item Let $a \neq 0$ be a scalar. Prove that $\{\vec{w}_1, \vec{w}_2, \ldots, \vec{w}_{m}\}$ is a linearly independent set of vectors if and only if the set $\{a\vec{w}_1, \vec{w}_2, \ldots, \vec{w}_{m}\}$ is also linearly independent.
\item Use Parts (a) and (b) to explain why the number of linearly independent rows in a matrix is not changed by row operations.
\item Use Part (c) to explain why the rank of a matrix is equal to the number of linearly independent rows.
\end{enumerate}

\begin{sol}
Write your solution here.
\end{sol}
\clearpage

\section*{Optional Problems}

\item Prove any linearly  independent  set of vectors in $\R^n$ can be expanded to a basis for $\R^n$.
\item Prove any set of vectors whose span is $\R^n$ contains a basis for $\R^n$.
\item Prove that the set of solutions to any matrix equation $A\vec{x} = \vec{b}$ is a set of all affine combinations of some set of vectors.


\end{enumerate}


\end{document}