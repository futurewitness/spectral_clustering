\documentclass[11pt]{article}

\usepackage{epsfig}
\usepackage{amsfonts}
\usepackage{amssymb}
\usepackage{amstext}
\usepackage{amsmath}
\usepackage{xspace}
\usepackage{theorem}
\usepackage{graphicx}
\usepackage{tikz}
\usepackage{pgfplots}

% \usepackage{layout}% if you want to see the layout parameters
% and now use \layout command in the body

% This is the stuff for normal spacing
\makeatletter
 \setlength{\textwidth}{6in}
 \setlength{\oddsidemargin}{0in}
 \setlength{\evensidemargin}{0.5in}
 \setlength{\topmargin}{0in}
 \setlength{\textheight}{9in}
 \setlength{\headheight}{0pt}
 \setlength{\headsep}{0pt}
 \setlength{\marginparwidth}{59pt}

 \setlength{\parindent}{0pt}
 \setlength{\parskip}{5pt plus 1pt}
 \setlength{\theorempreskipamount}{5pt plus 1pt}
 \setlength{\theorempostskipamount}{0pt}
 \setlength{\abovedisplayskip}{8pt plus 3pt minus 6pt}


\newenvironment{proof}{{\bf Proof:  }}{\hfill\rule{2mm}{2mm}}
\newenvironment{proofof}[1]{{\bf Proof of #1:  }}{\hfill\rule{2mm}{2mm}}
\newenvironment{proofofnobox}[1]{{\bf#1:  }}{}
\newenvironment{example}{{\bf Example:  }}{\hfill\rule{2mm}{2mm}}


\newtheorem{theorem}{Theorem}
\newtheorem{lemma}[theorem]{Lemma}
\newtheorem{proposition}[theorem]{Proposition}
\newtheorem{claim}[theorem]{Claim}
\newtheorem{corollary}[theorem]{Corollary}
\newtheorem{definition}[theorem]{Definition}

% math notation
\newcommand{\R}{\ensuremath{\mathbb R}}
\newcommand{\Z}{\ensuremath{\mathbb Z}}
\newcommand{\N}{\ensuremath{\mathbb N}}
\newcommand{\F}{\ensuremath{\mathbb F}}
\newcommand{\C}{\ensuremath{\mathbb C}}

\newcommand{\size}[1]{\ensuremath{\left|#1\right|}}
\newcommand{\ceil}[1]{\ensuremath{\left\lceil#1\right\rceil}}
\newcommand{\floor}[1]{\ensuremath{\left\lfloor#1\right\rfloor}}


% anupam's abbreviations
\newcommand{\mnote}[1]{\normalmarginpar \marginpar{\tiny #1}}


% vectors
\renewcommand{\vec}[1]{\ensuremath{\mathbf{#1}}}

\newenvironment{sol}
    {\emph{Solution:}
    }


%%%%%%%%%%%%%%%%%%%%%%%%%%%%%%%%%%%%%%%%%%%%%%%%%%%%%%%%%%%%%%%%%%%%%%%%%%%
% Document begins here %%%%%%%%%%%%%%%%%%%%%%%%%%%%%%%%%%%%%%%%%%%%%%%%%%%%
%%%%%%%%%%%%%%%%%%%%%%%%%%%%%%%%%%%%%%%%%%%%%%%%%%%%%%%%%%%%%%%%%%%%%%%%%%%


\newcommand{\headings}{
\large{\textbf{YOUR NAME GOES HERE \hfill 21-241 Fall 2019}\\
\textbf{Homework 10 \hfill Due Friday, November 8}}\\
\rule[0.1in]{\textwidth}{0.01in}
%\thispagestyle{empty}
}

\pagestyle{empty}

\begin{document}

\headings



\begin{enumerate}
\section*{Required Problems}
\item Let $x$ and $y$ be complex numbers, and let $\overline{x}$ denote the complex conjugate of $x$. 
\begin{enumerate}
\item  Prove that $(\overline{x})(\overline{y}) = \overline{xy}$.
\item Prove that $\overline{x} + \overline{y} = \overline{x+y}$.
\end{enumerate}


 \begin{sol}
Write your solution here.
\end{sol}
\clearpage


\item (Strang 6.4.1) Which of these matrices $ASB$ will be symmetric with eigenvalues $1$ and $-1$?
\[ 
\begin{bmatrix} 1 & 0 \\ 1 & 1 \end{bmatrix} \begin{bmatrix} 1 & 0 \\ 0 & -1 \end{bmatrix} \begin{bmatrix} 1 & 1 \\ 0 & 1 \end{bmatrix} \quad
 \begin{bmatrix} 1 & 0 \\ 1 & 1 \end{bmatrix} \begin{bmatrix} 1 & 0 \\ 0 & -1 \end{bmatrix} \begin{bmatrix} 1 & 0 \\ -1 & 1 \end{bmatrix} \quad
  \begin{bmatrix} 0 & -1 \\ 1 & 0 \end{bmatrix} \begin{bmatrix} 1 & 0 \\ 0 & -1 \end{bmatrix} \begin{bmatrix} 0 & 1 \\ -1 & 0 \end{bmatrix}
\]
$B=A^T$ doesn't do it.  $B=A^{-1}$ doesn't do it. $B= \quad = \quad$ will succeed.  So $B$ must be an \hspace{1in} matrix.

 \begin{sol}
Write your solution here.
\end{sol}
\clearpage


\item (Strang 6.4.13) Write $S$ and $B$ in the form $\lambda_1 \vec{x}_1\vec{x}_1^T + \lambda_2 \vec{x}_2\vec{x}_2^T$ of the Spectral theorem $Q \Lambda Q^T$:

\[S = \begin{bmatrix} 3 & 1 \\ 1 & 3 \end{bmatrix} \quad B = \begin{bmatrix} 9 & 12 \\ 12 & 16 \end{bmatrix}  \quad \text{ (keep }
\lVert\vec{x}_1 \rVert = \lVert\vec{x}_2 \rVert = 1).\]

 \begin{sol}
Write your solution here.
\end{sol}
\clearpage


\item (Strang 6.5.7) Test to see if $A^TA$ is positive definite in each case: $A$ needs independent columns.

\[A = \begin{bmatrix} 1 & 2 \\ 0 & 3 \end{bmatrix} \text{ and }
A = \begin{bmatrix} 1 & 1 \\ 1 & 2 \\ 2 & 1 \end{bmatrix} \text{ and }
A = \begin{bmatrix} 1 & 1 & 2 \\ 1 & 2 & 1 \end{bmatrix}
\]

 \begin{sol}
Write your solution here.
\end{sol}
\clearpage



\item (Strang 6.5.22) From $S = Q \Lambda Q^T$ compute the positive definite symmetric square root $Q \sqrt{\Lambda} Q^T$ of each matrix.  Check that this square root gives $A^TA = S$:

\[S = \begin{bmatrix} 5 & 4 \\ 4 & 5 \end{bmatrix} \quad \text{ and } \quad
S = \begin{bmatrix} 10 & 6 \\ 6 & 10  \end{bmatrix}. 
\]


 \begin{sol}
Write your solution here.
\end{sol}
\clearpage


\item (Strang 7.2.1) Find the eigenvalues of these matrices.  Then find singular values from $A^TA$:

\[A = \begin{bmatrix} 0 & 4 \\ 0 & 0 \end{bmatrix} \quad
A = \begin{bmatrix} 0 & 4 \\ 1 & 0 \end{bmatrix}.
\]

For each $A$, construct $V$ from the eigenvectors  of $A^TA$ and $U$ from the eigenvectors of $AA^T$.  Check that $A = U \Sigma V^T$.


 \begin{sol}
Write your solution here.
\end{sol}
\clearpage


\item (Strang 7.2.4) Compute $A^TA$ and $AA^T$ and their eigenvalues and unit eigenvectors for $V$ and $U$.

\[ \textbf{Rectangular matrix} \quad A = \begin{bmatrix} 1 & 1 & 0 \\ 0 & 1 & 1 \end{bmatrix}.\]
Check $AV = U \Sigma$ (this decides $\pm$ signs in $U$). $\Sigma $ has the same shape as $A$: $2 \times 3$.


 \begin{sol}
Write your solution here.
\end{sol}
\clearpage


\item (Strang 7.2.5)
\begin{enumerate}
\item  The row space of $A = \begin{bmatrix} 1 & 1  \\ 3 & 3  \end{bmatrix}$ is 1-dimensional.  Find $\vec{v}_1$ in the row space and $\vec{u}_1$ in the column space.  What is $\sigma_1$?  Why is there no $\sigma_2$?
\item Choose $\vec{v}_2$ and $\vec{u}_2$ in $V$ and $U$.  Then $A = U \Sigma V^T = \vec{u}_1 \sigma_1\vec{v}_1^T$ (one term only).
\end{enumerate}


 \begin{sol}
Write your solution here.
\end{sol}
\clearpage



\section*{Optional Problems}

\item Let $p$ be a polynomial with real coefficients.  Prove that $p(\overline{x}) = 0$ if and only if $p(x) = 0$.  Conclude that every real polynomial with odd degree has at least one real root.


\item Prove that every entry on the diagonal of a positive definite matrix must be positive.

\end{enumerate}



\end{document}