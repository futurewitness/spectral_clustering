\documentclass[11pt]{article}

\usepackage{epsfig}
\usepackage{amsfonts}
\usepackage{amssymb}
\usepackage{amstext}
\usepackage{amsmath}
\usepackage{xspace}
\usepackage{theorem}
\usepackage{graphicx}
\usepackage{tikz}
\usepackage{pgfplots}

% \usepackage{layout}% if you want to see the layout parameters
% and now use \layout command in the body

% This is the stuff for normal spacing
\makeatletter
 \setlength{\textwidth}{6in}
 \setlength{\oddsidemargin}{0in}
 \setlength{\evensidemargin}{0.5in}
 \setlength{\topmargin}{0in}
 \setlength{\textheight}{9in}
 \setlength{\headheight}{0pt}
 \setlength{\headsep}{0pt}
 \setlength{\marginparwidth}{59pt}

 \setlength{\parindent}{0pt}
 \setlength{\parskip}{5pt plus 1pt}
 \setlength{\theorempreskipamount}{5pt plus 1pt}
 \setlength{\theorempostskipamount}{0pt}
 \setlength{\abovedisplayskip}{8pt plus 3pt minus 6pt}


\newenvironment{proof}{{\bf Proof:  }}{\hfill\rule{2mm}{2mm}}
\newenvironment{proofof}[1]{{\bf Proof of #1:  }}{\hfill\rule{2mm}{2mm}}
\newenvironment{proofofnobox}[1]{{\bf#1:  }}{}
\newenvironment{example}{{\bf Example:  }}{\hfill\rule{2mm}{2mm}}


\newtheorem{theorem}{Theorem}
\newtheorem{lemma}[theorem]{Lemma}
\newtheorem{proposition}[theorem]{Proposition}
\newtheorem{claim}[theorem]{Claim}
\newtheorem{corollary}[theorem]{Corollary}
\newtheorem{definition}[theorem]{Definition}

% math notation
\newcommand{\R}{\ensuremath{\mathbb R}}
\newcommand{\Z}{\ensuremath{\mathbb Z}}
\newcommand{\N}{\ensuremath{\mathbb N}}
\newcommand{\F}{\ensuremath{\mathbb F}}
\newcommand{\C}{\ensuremath{\mathbb C}}

\newcommand{\size}[1]{\ensuremath{\left|#1\right|}}
\newcommand{\ceil}[1]{\ensuremath{\left\lceil#1\right\rceil}}
\newcommand{\floor}[1]{\ensuremath{\left\lfloor#1\right\rfloor}}


% anupam's abbreviations
\newcommand{\mnote}[1]{\normalmarginpar \marginpar{\tiny #1}}


% vectors
\renewcommand{\vec}[1]{\ensuremath{\mathbf{#1}}}

\newenvironment{sol}
    {\emph{Solution:}
    }


%%%%%%%%%%%%%%%%%%%%%%%%%%%%%%%%%%%%%%%%%%%%%%%%%%%%%%%%%%%%%%%%%%%%%%%%%%%
% Document begins here %%%%%%%%%%%%%%%%%%%%%%%%%%%%%%%%%%%%%%%%%%%%%%%%%%%%
%%%%%%%%%%%%%%%%%%%%%%%%%%%%%%%%%%%%%%%%%%%%%%%%%%%%%%%%%%%%%%%%%%%%%%%%%%%


\newcommand{\headings}{
\large{\textbf{YOUR NAME GOES HERE \hfill 21-241 Fall 2019}\\
\textbf{Homework 8 \hfill Due Friday, October 25}}\\
\rule[0.1in]{\textwidth}{0.01in}
%\thispagestyle{empty}
}

\pagestyle{empty}

\begin{document}

\headings

\begin{enumerate}
\section*{Required Problems}


\item (Strang 6.1.3) Compute the eigenvalues and eigenvectors of $A$ and $A^{-1}$. %Check the trace!

\[ A = \begin{bmatrix} 0 & 2 \\ 1 & 1 \end{bmatrix} \text{ and } A^{-1} = \begin{bmatrix} -1/2 & 1 \\ 1/2 & 0 \end{bmatrix}.\]
$A^{-1}$ has the \hspace{1in} eigenvectors as $A$.  When $A$ has eigenvalues $\lambda_1$ and $\lambda_2$, its inverse has eigenvalues \hspace{1in}.


 \begin{sol}
Write your solution here.
\end{sol}
\clearpage


\item (Strang 6.1.12) Find three eigenvectors for this matrix $P$ (Projection matrices have $\lambda = 1$ and $0$.):

\[\textbf{Projection matrix} \hspace{.5in} P = \begin{bmatrix} .2 & .4 & 0 \\ .4 & .8 & 0 \\ 0 & 0 & 1  \end{bmatrix}\]

If two eigenvectors share the same $\lambda$, so do all of their linear combinations.  Find an eigenvector of $P$ with no zero components.


 \begin{sol}
Write your solution here.
\end{sol}
\clearpage

\item (Strang 6.1.16) The determinant of $A$ equals the product $\lambda_1\lambda_2 \cdots \lambda_n$.  Start with the polynomial $\det(A-\lambda I)$ separated into its $n$ factors (always possible). Then set $\lambda = 0$:
\[\det (A - \lambda I) = (\lambda_1 - \lambda)(\lambda_2 - \lambda) \cdots (\lambda_n - \lambda) \text{ so } \det A = \]
Check this rule in Example 1 where the Markov matrix has $\lambda = 1$ and $1/2$.


 \begin{sol}
Write your solution here.
\end{sol}
\clearpage

\item (Strang 6.1.25) Suppose $A$ and $B$ have the same eigenvalues $\lambda_1, \ldots, \lambda_n$ with the same independent eigenvectors $\vec{x}_1, \ldots, \vec{x}_n$.  Then $A=B$.  Reason: Any vector $\vec{x}$ is a linear combination $c_1 \vec{x}_1 + \cdots + c_n \vec{x}_n$.  What is $A\vec{x}$?  What is $B\vec{x}$?



 \begin{sol}
Write your solution here.
\end{sol}
\clearpage


\item (Strang 6.2.16) Find $\Lambda$ and $X$ to diagonalize $A= \begin{bmatrix} .6 & .9 \\ .4 & .1 \end{bmatrix}$ (in Strang 6.2.15).  What is the limit of $\Lambda^k$ as $k \to \infty$?  What is the limit of $X\Lambda^kX^{-1}$?  In the columns of this limiting matrix you see the \hspace{1in}.



 \begin{sol}
Write your solution here.
\end{sol}
\clearpage


\item (Strang 6.2.29) Suppose the same $X$ diagonalizes both $A$ and $B$. They have the same eigenvectors in $A = X \Lambda_1 X^{-1}$ and $B = X \Lambda_2 X^{-1}$.  Prove that $AB = BA$.



 \begin{sol}
Write your solution here.
\end{sol}
\clearpage

\item Let $A$ be a matrix, and $\vec{x}$ and $\vec{y}$ be eigenvectors for $A$. Prove or disprove each of the following statements.
\begin{enumerate}
\item For all scalars $c \neq 0$, the vector $c\vec{x}$ is an eigenvector for $A$.
\item For all integers $k \ge 1$, $x$ is an eigenvector for $A^k$.
\item The vector $\vec{x} + \vec{y}$ is always an eigenvector for $A$.
\end{enumerate}



 \begin{sol}
Write your solution here.
\end{sol}
\clearpage


\section*{Optional Problems}

\item Prove that if $A_1$ is similar to $A_2$ and $A_2$ is similar to $A_3$, then $A_1$ is similar to $A_3$.

\item Prove or disprove:
\begin{enumerate}
\item If $\vec{x}$ is an eigenvector for $A$ and $B$, then $\vec{x}$ is an eigenvector for $AB$ and $BA$.
\item If $\lambda$ is an eigenvalue for $A$ and $B$, then $\lambda^2$ is an eigenvalue for $AB$ and $BA$.
\end{enumerate}

\item  List all matrices that are similar to the identity matrix.

\item Prove that the eigenvalues of a triangular matrix are the entries on the diagonal.

\item The trace of a matrix is the sum of the diagonal entries.  Prove that the sum of the eigenvalues is equal to the trace.

\item Suppose $\vec{x}_1$ and $\vec{x}_2$ are eigenvectors for $A$ with eigenvalues $\lambda_1$ and $\lambda_2$.  Under what conditions on $\lambda_1$ and $\lambda_2$ is $\vec{x}_1 + \vec{x}_2$ an eigenvector for $A$?

\item We have seen how it is possible to find eigenvalues and eigenvectors of a matrix by finding roots of its characteristic polynomial.  In this problem you will show how to do the reverse:  You can find the roots of a polynomial by finding the eigenvectors of its ``companion matrix.'' 
Let $p$ be the degree $n$ polynomial $p(z) = c_0 + c_1z + c_2z^2 + \cdots + z^n$. 

Note that the coefficient of $z^n$ is 1.

Define the companion matrix for $p$ to be the $n \times n$ matrix

\[C=\begin{bmatrix}
0 & 1 & 0 & \dots & 0 \\
0 & 0 & 1 & \dots & 0 \\
0 & \ddots & \ddots & \ddots & \vdots \\
\vdots & \vdots & \ddots & 0 & 1 \\
-c_0 & -c_1 & \dots & -c_{n-2} & -c_{n-1}
\end{bmatrix}.\]

\begin{enumerate}
\item Show that $\det (C- \lambda I) = p(\lambda)$. 
\item Prove that $z$ is a root of $p$ if and only if it is an eigenvalue of $C$ with eigenvector $(1, z, z^2, \ldots, z^{n-1})$.
\item Explain how to determine the roots of any degree $n$ polynomial (even if its leading coefficient is not 1) if you know how to find eigenvectors for a matrix. [This is actually how some polynomial solvers proceed: Rather than solving the polynomial they instead find the eigenvectors of its companion matrix. ]

\end{enumerate}



\item Write Julia code to recursively calculate the determinant of any $n \times n$ matrix using cofactors.

\end{enumerate}





\end{document}