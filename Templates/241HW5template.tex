\documentclass[11pt]{article}

\usepackage{epsfig}
\usepackage{amsfonts}
\usepackage{amssymb}
\usepackage{amstext}
\usepackage{amsmath}
\usepackage{xspace}
\usepackage{theorem}
\usepackage{graphicx}
\usepackage{tikz}
\usepackage{pgfplots}

% \usepackage{layout}% if you want to see the layout parameters
% and now use \layout command in the body

% This is the stuff for normal spacing
\makeatletter
 \setlength{\textwidth}{6in}
 \setlength{\oddsidemargin}{0in}
 \setlength{\evensidemargin}{0.5in}
 \setlength{\topmargin}{0in}
 \setlength{\textheight}{9in}
 \setlength{\headheight}{0pt}
 \setlength{\headsep}{0pt}
 \setlength{\marginparwidth}{59pt}

 \setlength{\parindent}{0pt}
 \setlength{\parskip}{5pt plus 1pt}
 \setlength{\theorempreskipamount}{5pt plus 1pt}
 \setlength{\theorempostskipamount}{0pt}
 \setlength{\abovedisplayskip}{8pt plus 3pt minus 6pt}


\newenvironment{proof}{{\bf Proof:  }}{\hfill\rule{2mm}{2mm}}
\newenvironment{proofof}[1]{{\bf Proof of #1:  }}{\hfill\rule{2mm}{2mm}}
\newenvironment{proofofnobox}[1]{{\bf#1:  }}{}
\newenvironment{example}{{\bf Example:  }}{\hfill\rule{2mm}{2mm}}


\newtheorem{theorem}{Theorem}
\newtheorem{lemma}[theorem]{Lemma}
\newtheorem{proposition}[theorem]{Proposition}
\newtheorem{claim}[theorem]{Claim}
\newtheorem{corollary}[theorem]{Corollary}
\newtheorem{definition}[theorem]{Definition}

% math notation
\newcommand{\R}{\ensuremath{\mathbb R}}
\newcommand{\Z}{\ensuremath{\mathbb Z}}
\newcommand{\N}{\ensuremath{\mathbb N}}
\newcommand{\F}{\ensuremath{\mathbb F}}
\newcommand{\C}{\ensuremath{\mathbb C}}

\newcommand{\size}[1]{\ensuremath{\left|#1\right|}}
\newcommand{\ceil}[1]{\ensuremath{\left\lceil#1\right\rceil}}
\newcommand{\floor}[1]{\ensuremath{\left\lfloor#1\right\rfloor}}


% anupam's abbreviations
\newcommand{\mnote}[1]{\normalmarginpar \marginpar{\tiny #1}}


% vectors
\renewcommand{\vec}[1]{\ensuremath{\mathbf{#1}}}

\newenvironment{sol}
    {\emph{Solution:}
    }


%%%%%%%%%%%%%%%%%%%%%%%%%%%%%%%%%%%%%%%%%%%%%%%%%%%%%%%%%%%%%%%%%%%%%%%%%%%
% Document begins here %%%%%%%%%%%%%%%%%%%%%%%%%%%%%%%%%%%%%%%%%%%%%%%%%%%%
%%%%%%%%%%%%%%%%%%%%%%%%%%%%%%%%%%%%%%%%%%%%%%%%%%%%%%%%%%%%%%%%%%%%%%%%%%%


\newcommand{\headings}{
\large{\textbf{YOUR NAME GOES HERE \hfill 21-241 Fall 2019}\\
\textbf{Homework 5 \hfill Due Friday, September 27}}\\
\rule[0.1in]{\textwidth}{0.01in}
%\thispagestyle{empty}
}

\pagestyle{empty}

\begin{document}

\headings

\begin{enumerate}
\section*{Required Problems}
\item (Strang 4.1.1) Construct any 2 by 3 matrix of rank one.  Copy Figure 4.2 and put one vector in each subspace (and put two in the nullspace).  Which vectors are orthogonal? [Use a different matrix than the one we used in class.]

 \begin{sol}
Write your solution here.
\end{sol}
\clearpage

\item (Strang 4.1.11) Draw Figure 4.2 to show each subspace correctly for 
\[A = \begin{bmatrix} 1 & 2 \\ 3 & 6 \end{bmatrix} \text{ and } B = \begin{bmatrix} 1 & 0 \\ 3 & 0 \end{bmatrix}.  \]

 \begin{sol}
Write your solution here.
\end{sol}
\clearpage

\item (Strang 4.1.17) If $S$ is the subspace of $\R^3$ containing only the zero vector, what is $S^\perp$? If $S$ is spanned by $(1,1,1)$, what is $S^\perp$?  If $S$ is spanned by $(1,1,1)$ and $(1,1,-1)$, what is a basis for $S^\perp$?

 \begin{sol}
Write your solution here.
\end{sol}
\clearpage

\item (from Strang 4.2.6) Project $\vec{b} = (1,0,0)$ onto the lines through $\vec{a}_1 = (-1, 2, 2)$, $\vec{a}_2 = (2,2,-1)$, and $\vec{a}_3 = (2,-1,2)$.  Add up the three projections $\vec{p}_1 + \vec{p}_2 + \vec{p}_3$. [What do you notice?  Note the $\vec{a}$'s are orthogonal.]

 \begin{sol}
Write your solution here.
\end{sol}
\clearpage

\item (from Strang 4.2.8) Project the vector $\vec{b} = (1,1)$ onto the lines through $\vec{a}_1 = (1,0)$ and $\vec{a}_2 = (1,2)$. Draw the projections $\vec{p}_1$ and $\vec{p}_2$ and add $\vec{p}_1 + \vec{p}_2$. The projections do not add to $\vec{b}$ because the $\vec{a}$'s are not orthogonal.

 \begin{sol}
Write your solution here.
\end{sol}
\clearpage

\item (Strang 4.2.21) Multiply the matrix $P= A(A^TA)^{-1}A^T$ by itself.  Cancel to prove that $P^2 = P$.  Explain why $P(P\vec{b})$ always equals $P\vec{b}$: The vector $P\vec{b}$ is in the column space of $A$ so its projection onto that column space is ...

 \begin{sol}
Write your solution here.
\end{sol}
\clearpage

\item (Strang 4.3.1. See Page 228 for a picture.) With $\vec{b} = 0, 8, 8, 20$ at $t= 0, 1, 3, 4$, set up and solve the normal equations $A^TA\hat{\vec{x}}= A^T\vec{b}$. For the best straight line in Figure 4.9a, find its four heights $p_i$ and four errors $e_i$.  What is the minimum value $E= e_1^2 + e_2^2 + e_3^2 + e_4^2$?

 \begin{sol}
Write your solution here.
\end{sol}
\clearpage

\item (Strang 4.3.9) For the closest parabola $b=C + Dt + Et^2$ to the same four points (as in Strang 4.3.1), write down the unsolvable equations $A \vec{x} = \vec{b}$ in three unknowns $\vec{x} = (C, D, E)$. Set up the three normal equations $A^TA\hat{\vec{x}} = A^T\vec{b}$ (solution not required).  In Figure 4.9a you are now fitting a parabola to four points.  What is happening in Figure 4.9b?

 \begin{sol}
Write your solution here.
\end{sol}
\clearpage
\section*{Optional Problems}



\item (Strang 4.2.2) \textit{Draw} the projection of $\vec{b}$ onto $\vec{a}$ and also compute it from $\vec{p} = \hat{\vec{x}}\vec{a}$:

\[ \text{(a) } \vec{b} =  \begin{bmatrix} \cos \theta \\ \sin \theta \end{bmatrix} \text{ and } \vec{a} = \begin{bmatrix} 1 \\ 0  \end{bmatrix}  \quad
\text{ (b) } \vec{b} =  \begin{bmatrix} 1 \\ 1 \end{bmatrix} \text{ and } \vec{a} = \begin{bmatrix} 1 \\ -1  \end{bmatrix} \]


\item Let $\vec{v}, \vec{w}, \vec{x} \in \R^n$, $c, \in \R$. 
\begin{enumerate}
\item Prove that $(c\vec{v})^T\vec{w}= c(\vec{v}^T\vec{w})$.
\item Prove that $(\vec{v} + \vec{x})^T\vec{w}= \vec{v}^T\vec{w} + \vec{x}^T\vec{w}$.
\end{enumerate}

\item Let $\vec{v}, \vec{w} \in \R^n$. Prove the \textit{Triangle inequality}:
\[ \lVert \vec{v} + \vec{w} \rVert \le \lVert \vec{v} \rVert + \lVert \vec{w} \rVert.\]
What does this inequality have to do with triangles?

\item Let $\vec{v}, \vec{w} \in \R^n$. Prove the \textit{Cauchy-Schwarz inequality}:
\[  | \vec{v} \cdot \vec{w} |\le \lVert \vec{v} \rVert  \lVert \vec{w} \rVert.\]

\item Let $A$ be an $m \times n$ matrix.  Prove that if $V= \{\vec{v}_1, \ldots, \vec{v}_r\}$ is a basis for $C(A^T)$ and $W= \{\vec{w}_1, \ldots, \vec{w}_{n-r}\}$ is a basis for $N(A)$, then $V \cup W$ is a basis for $\R^n$.

\item Let $\vec{v},\vec{w} \in \R^n$.
\begin{enumerate}
\item Prove the Pythagorean theorem.
\item Prove that 
\[ \lVert \vec{v} \rVert^2 + \lVert \vec{w} \rVert^2 = \lVert \vec{v} - \vec{w} \rVert^2\]
if and only if $\vec{v} \perp \vec{w}$.
\end{enumerate}


 \item Let $\vec{v} \in \R^n$. Prove that if $\vec{v}^T\vec{v} = 0$ then $\vec{v} = \vec{0}$. Show that this is not true in $\C^n$.  Hint: Consider the vector $\vec{v} = (1,i) \in \C^2$.


\end{enumerate}


\end{document}