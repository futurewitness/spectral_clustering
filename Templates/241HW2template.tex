\documentclass[11pt]{article}

\usepackage{epsfig}
\usepackage{amsfonts}
\usepackage{amssymb}
\usepackage{amstext}
\usepackage{amsmath}
\usepackage{xspace}
\usepackage{theorem}
\usepackage{graphicx}
% \usepackage{layout}% if you want to see the layout parameters
% and now use \layout command in the body

% This is the stuff for normal spacing
\makeatletter
 \setlength{\textwidth}{6in}
 \setlength{\oddsidemargin}{0in}
 \setlength{\evensidemargin}{0.5in}
 \setlength{\topmargin}{0in}
 \setlength{\textheight}{9in}
 \setlength{\headheight}{0pt}
 \setlength{\headsep}{0pt}
 \setlength{\marginparwidth}{59pt}

 \setlength{\parindent}{0pt}
 \setlength{\parskip}{5pt plus 1pt}
 \setlength{\theorempreskipamount}{5pt plus 1pt}
 \setlength{\theorempostskipamount}{0pt}
 \setlength{\abovedisplayskip}{8pt plus 3pt minus 6pt}


\newenvironment{proof}{{\bf Proof:  }}{\hfill\rule{2mm}{2mm}}
\newenvironment{proofof}[1]{{\bf Proof of #1:  }}{\hfill\rule{2mm}{2mm}}
\newenvironment{proofofnobox}[1]{{\bf#1:  }}{}
\newenvironment{example}{{\bf Example:  }}{\hfill\rule{2mm}{2mm}}


\newtheorem{theorem}{Theorem}
\newtheorem{lemma}[theorem]{Lemma}
\newtheorem{proposition}[theorem]{Proposition}
\newtheorem{claim}[theorem]{Claim}
\newtheorem{corollary}[theorem]{Corollary}
\newtheorem{definition}[theorem]{Definition}

% math notation
\newcommand{\R}{\ensuremath{\mathbb R}}
\newcommand{\F}{\ensuremath{\mathbb F}}

% vectors
\renewcommand{\vec}[1]{\ensuremath{\mathbf{#1}}}

\newenvironment{sol}
    {\emph{Solution:}
    }

%%%%%%%%%%%%%%%%%%%%%%%%%%%%%%%%%%%%%%%%%%%%%%%%%%%%%%%%%%%%%%%%%%%%%%%%%%%
% Document begins here %%%%%%%%%%%%%%%%%%%%%%%%%%%%%%%%%%%%%%%%%%%%%%%%%%%%
%%%%%%%%%%%%%%%%%%%%%%%%%%%%%%%%%%%%%%%%%%%%%%%%%%%%%%%%%%%%%%%%%%%%%%%%%%%


\newcommand{\headings}{
\large{\textbf{YOUR NAME GOES HERE \hfill 21-241 Fall 2019}\\
\textbf{Homework 2 \hfill Due Friday, September 6}}\\
\rule[0.1in]{\textwidth}{0.01in}
%\thispagestyle{empty}
}

\pagestyle{empty}

\begin{document}

\headings

\begin{enumerate}
\section*{Required Problems}

\item (Strang 2.3.10) 
\begin{enumerate}
\item What 3 by 3 matrix $E_{13}$ will add row 3 to row 1?
\item What matrix adds row 1 to row 3 and \textit{at the same time} row 3 to row 1?
\item What matrix adds row 1 to row 3 and \textit{then} adds row 3 to row 1?
\end{enumerate}


\begin{sol}
Write your solution here.
\end{sol}

\item (Strang 2.6.5) What matrix $E$ puts $A$ into a triangular form $EA=U$?  Multiply by $E^{-1} = L$ to factor $A$ into $LU$:

\[A = \begin{bmatrix} 2 & 1 & 0 \\ 0 & 4 & 2 \\ 6 & 3 & 5 \end{bmatrix} \]



\begin{sol}
Write your solution here.
\end{sol}


\item (Strang 2.7.22) Find the $PA = LU$ factorizations (and check them) for 
\[A = \begin{bmatrix} 0 & 1 & 1 \\ 1 & 0 & 1 \\ 2 & 3 & 4 \end{bmatrix} \text{ and } A=  \begin{bmatrix} 1 & 2 & 0 \\ 2 & 4 & 1 \\ 1 & 1 & 1 \end{bmatrix}\]


\begin{sol}
Write your solution here.
\end{sol}

\item (Strang 2.6.10) Compute $L$ and $U$ for the symmetric matrix $A$:

\[ A = \begin{bmatrix} a& a& a&a \\ a& b & b & b \\ a& b & c & c \\ a & b & c& d \end{bmatrix} \]


\begin{sol}
Write your solution here.
\end{sol}

\item (Strang 2.6.13) Solve $L\vec{c} = \vec{b}$ to find $\vec{c}$. Then solve $U\vec{x} = \vec{c}$ to find $\vec{x}$.  What was $A$?

\[ L = \begin{bmatrix} 1 & 0 & 0 \\ 1 & 1 & 0 \\ 1 & 1 & 1 \end{bmatrix} \text{ and } U= \begin{bmatrix} 1 & 1 & 1 \\ 0 & 1 & 1 \\ 0 & 0 & 1\end{bmatrix} \text{ and } \vec{b} = \begin{bmatrix} 4 \\ 5 \\ 6 \end{bmatrix}\]


\begin{sol}
Write your solution here.
\end{sol}

\item An \textbf{affine combination} of a set of vectors $\vec{v}_1, \vec{v_2}, \ldots, \vec{v_n}$ is a linear combination $a_1\vec{v}_1 + a_2 \vec{v_2} + \dots + a_n\vec{v_n}$ where $\sum_{i=1}^n a_i = 1$, i.e. the sum of the coefficients is equal to 1.
\begin{enumerate}
\item Suppose $A$ is a matrix and $\vec{b}$, $\vec{x}$, and $\vec{y}$ are vectors such that $\vec{x} \neq \vec{y}$, $A\vec{x} = \vec{b}$, and $A\vec{y} = \vec{b}$.  Prove that if $\vec{v}$ is an affine combination of $\vec{x}$ and $\vec{y}$, then $A\vec{v}=b$.  

You may use the facts that if $c$ is a scalar, $\vec{x}$ and $\vec{y}$ are vectors, and $A$ is a matrix, then
\begin{itemize}
\item  $A(c\vec{x})) = c(A\vec{x})$, and 
\item $A(\vec{x} +  \vec{y}) =  A\vec{x} +  A\vec{y}$.
\end{itemize}

\item Use Part (a) to explain why a system of linear equations cannot have exactly two solutions
\item Use Julia to generate some random vectors in $\R^2$, and plot affine combinations of them.  What geometric object is formed by the set of all affine combinations of two vectors?
\end{enumerate}


\begin{sol}
Write your solution here.
\end{sol}

\item A square matrix $A$ is \textbf{symmetric} if for all $i$ and $j$, $A_{ij} = A_{ji}$. Prove that for any $m \times n$ matrix $A$, $AA^T$ is a symmetric $m \times m$ matrix.


\begin{sol}
Write your solution here.
\end{sol}

\section*{Optional Problems}

\item Prove that the identity matrix is unique, i.e. if some $n \times n$ matrix $A$ has the property that $A\vec{x}=\vec{x}$ for all $\vec{x} \in \R^n$, then $A=I$.

\item Explain why it is always possible to change one element of a nonsingular matrix so that the result is singular.  Give an example to show the converse is not true, i.e. give an example of a singular matrix where no matter what element is changed, the result is still singular.

\item Coding problems
\begin{enumerate}
\item Write Julia code that will return the $LU$ decomposition of a square nonsingular matrix $A$.
\item Write Julia code that, given $b$ and a $PA=LU$ decomposition for the square matrix $A$, will return the solution $x$ to the matrix equation $Ax=b$.
\end{enumerate}


\end{enumerate}




\end{document}